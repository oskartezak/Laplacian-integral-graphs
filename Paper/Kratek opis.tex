\documentclass{article}
\usepackage[slovene]{babel}
\usepackage[utf8]{inputenc}
\usepackage[T1]{fontenc}
\usepackage{amsmath, amssymb}
\usepackage{amsthm}

\begin{document}

\title{Laplacian integral graphs}
\author{Oskar Težak, Sara Šega}

\maketitle

\section{Opis problema}

Poiskati želimo čim več enocikličnih in dvocikličnih grafov, katerih Laplaceove lastne vrednosti so cela števila. 
Za grafe nižjega reda to storimo z izčrpno metodo, za grafe višjega reda moramo uporabiti stohastično metodo. 
Najti želimo metodo oz. vzorec za generiranje Laplaceovih celoštevilskih grafov. \\ 
Naj bo $G(V,E)$ graf z množico vozlišč $V$, $ \left| V \right| = n $
in množico povezav $E$, $\left| E \right| = m. $ \\
Definiramo Laplaceovo matriko $ L = D - A $, kjer je $D$ $ n \times n $ diagonalna matrika, katere diagonalni členi so stopnje vozlišč grafa $G$, 
$A$ pa je $ n \times n $ matrika sosednosti. Elementi Laplaceove matrike $L$ so torej

\[
l_{i,j} = 
\begin{cases} 
    \deg(v_i), & \text{če } i = j, \\
    -1, & \text{če } i \neq j \text{ in } v_i \text{ je soseden z } v_j, \\
    0, & \text{sicer}.
\end{cases}
\]

Lastne vrednosti matrike $L$ so $ \lambda_1 \leq \lambda_2 \leq \dots \leq \lambda_n .$ Če so vse lastne
vrednosti pozitivne, pravimo, da je graf $G$ z matriko $L$ Laplaceov celoštevilski. 


\section{Potek dela}

\subsection{Generiranje enocikličnih Laplaceovih celoštevilskih grafov}
Najprej bova generirala povezane enociklične grafe (s pomočjo lastnosti, da imajo povezani enociklični grafi enako število vozlišč in povezav) in testirala, kateri izmed 
njih so Laplaceovi celoštevilski. To bova storila za manjše grafe ( do 10 vozlišč). V prikazanih grafih bova poskusila poiskati vzorec oz. njihovo skupno lastnost,
ki nam bi zagotovila, da je graf Laplaceov celoštevilski. \\

\subsection{Generiranje dvocikličnih Laplaceovih celoštevilskih grafov}
Na podoben način bova generirala manjše povezane dvociklične grafe (uporabila bova lastnost, da je število povezav za 1 večje od števila vozlišč) in spet poskusila
ugotoviti lastnost, ki je skupna vsem dvocikličnim Laplaceovim celoštevilskim grafom. \\

\subsection{Stohastično iskanje}
Za večje grafe bova uporabila stohastično metodo za generiranje enocikličnih in dvocikličnih grafov z večjim številom vozlišč. 
Cilj je poiskati nove Laplaceove celoštevilske grafe in preveriti, ali se vzorci, opaženi pri manjših grafih, ohranijo.\\

\subsection{Dodatna analiza}
Ker so Laplaceovi celoštevilski grafi redki, bova na koncu raziskala tudi grafe, ki imajo čim več celoštevilskih Laplaceovih lastnih vrednosti. 
Tako bova lahko bolje razumela, katere lastnosti grafov vplivajo na delno celoštevilsko Laplaceovo spektralno lastnost.


\end{document}