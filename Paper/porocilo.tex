\documentclass{article}
\usepackage[slovene]{babel}
\usepackage[utf8]{inputenc}
\usepackage[T1]{fontenc}
\usepackage{amsmath, amssymb}
\usepackage{amsthm}


\newcommand{\program}{Finančna matematika} % ime studijskega programa: Matematika/Finančna matematika
\newcommand{\imeavtorja}{Oskar Težak, Sara Šega} % ime avtorja
\newcommand{\imementorja}{doc. dr. Janoš Vidali} % akademski naziv in ime mentorja
\newcommand{\imesomentorja}{prof. dr. Riste Škrekovski}
\newcommand{\naslovdela}{Laplacian integral graphs}
\newcommand{\letnica}{2024} %letnica 


\begin{document}

%naslovnica

\thispagestyle{empty}
\noindent{\large
UNIVERZA V LJUBLJANI\\[1mm]
FAKULTETA ZA MATEMATIKO IN FIZIKO\\[5mm]
\program\ }
\vfill

\begin{center}{\large
\imeavtorja\\[2mm]
{\bf \naslovdela}\\[10mm]
Skupinski projekt\\[2mm]
Poročilo\\[1cm]
Mentorja: \imementorja, \\ \imesomentorja\\[2mm]}
\end{center}
\vfill

\noindent{\large
Ljubljana, december \letnica}
\pagebreak




\section{Opis problema}

Poiskati želimo čim več enocikličnih in dvocikličnih grafov, katerih Laplaceove lastne vrednosti so cela števila. 
Za grafe nižjega reda to storimo z izčrpno metodo, za grafe višjega reda moramo uporabiti stohastično metodo. 
Najti želimo metodo oz. vzorec za generiranje Laplaceovih celoštevilskih grafov. \\ 
Naj bo $G(V,E)$ graf z množico vozlišč $V, \left| V \right| = n
$ in množico povezav $E, \left| E \right| = m. $ \\
Definiramo Laplaceovo matriko $ L = D - A $, kjer je $D$ $ n \times n $ diagonalna matrika, katere diagonalni členi so stopnje vozlišč grafa $G$, 
$A$ pa je $ n \times n $ matrika sosednosti. Lastne vrednosti matrike $L$ so $ \lambda_1 \leq \lambda_2 \leq \dots \leq \lambda_n .$ Če so vse lastne
vrednosti pozitivne, pravimo, da je graf $G$ z matriko $L$ Laplaceov celoštevilski. 
%Značilnosti

\section{Potek}
Najprej sva napisala kodo za generiranje enocikličnih grafov, ki imajo relativno malo vozlišč. Uporabila sva dejstvo, da imajo povezani enociklični grafi enako
število povezav in vozlišč. Implementirala sva funkcijo, ki za vsak graf potem preveri, če je Laplaceov in Laplaceove tudi izriše. Funkcijo sva testirala
na grafih s 6, 7, 8, 9, 10 vozlišči. Za iskanje dvocikličnih grafov sva uporabila isto kodo, le da sva tokrat nastavila, da generirava grafe z $n$ vozlišči in $n+1$ povezavami.    

\[
\det(L - \lambda I) = 
\begin{vmatrix}
    n-k - \lambda &-1  &-1 &&\dots&&   -1  \\
    -1 & 2 - \lambda & -1&&&&  &    \\
    -1&  -1 & \ddots &\ddots&&&  & \\
    &  &\ddots  & 2 - \lambda &&&&    \\
    \vdots& &  &  &1 - \lambda &&  & \\
    &  &  &  &  &\ddots \\
    -1& &  &  &  & &1 - \lambda \\
\end{vmatrix}
\]

\end{document}